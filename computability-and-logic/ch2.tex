\section{Diagonalization} % (fold)
\label{sec:diagonalization}
\begin{theorem}[Cantor]
  The set of all sets of positive integers is not enumerable.
\end{theorem}
\begin{proof}
  Let $L = S_1, S_2, S_3, \ldots$ be an arbitrary list of sets of positive integers.
  Let $S'$ be a set of positive integers such that for each $n$, $n \in S'$ if and only if $n \not\in S_n$.

  Suppose there exists some $m$ such that $S' = S_m$, that is, $S'$ is on the list $L$ for some $m$.
  Then, by the construction of $S'$, $m \in S_m$ if and only if $m \not\in S_m$, absurd; whence there exists no such $m$, and $S'$ is not on the list.

  Since $L$ was arbitrary, this method works for any list $L$, in particular, the list $L$ of all sets of positive integers.
  Then no list enumerates the power set of positive integers; this completes the proof.
\end{proof}

\begin{theorem}
  The set of real numbers is not enumerable.
\end{theorem}
\begin{proof}
  Suppose there exists an enumeration $f\colon \mathbb N \to \mathbb R$.
  Then for each $x \in \mathbb R$ there exists an $n \in \mathbb N$ such that $f(n) = x$, as $f$ is surjective.
  Let $d_{n,0}$ be the digit before the decimal point of $f(n)$, and $d_{n,i}$ the $i$th digit of the decimal expansion of $f(n)$.
  For all $i$, let $e_i$ be a digit different from $d_{i,i}$.
  Now construct a real number $r$ whose decimal expansion is $e_0.e_1e_2\ldots$.
  For each $n \in \mathbb N$, we have $e_n \ne d_{n,n}$, and so $r \ne f(n)$.
  Then $r$ is a real number not in the image of $f$, whence $f$ isn't surjective.
\end{proof}

\begin{exercise}[2.1]
  The set of all subsets of an infinite enumerable set $A$ is nonenumerable.
\end{exercise}
\begin{proof}
  By an earlier exercise, there exists a correspondence $f\colon P \to A$.
  Then there is a correspondence $h\colon \mathcal P(A) \to \mathcal P(P)$ given by $\{a_i, a_j, \ldots\} \mapsto \{i, j, \ldots\}$.
  But by Cantor's theorem the power set of positive integers is nonenumerable, whence the power set of $A$ is nonenumerable also.
\end{proof}
\begin{proof}
  Suppose the power set of $A$ is enumerable.
  Then there exists a list $A_1,A_2,\ldots$ of subsets of $A$.
  Since each subset of an enumerable set is enumerable, we can write each subset $A_j$ of $A$ as a list $a_{j,1},a_{j,2},a_{j,3},\ldots$.
  Let $B$ be a subset such that for each $i$, $b_i \in B$ is different from $a_{i,i} \in A_i$.
  By the assumption, $B = A_n$ for some $n$.
  Examine the element $b_n$.
  By the construction, $b_n \ne a_{n,n} = b_n$, absurd; and it follows that the power set of $A$ is nonenumerable.
\end{proof}

\begin{exercise}[2.2]
  If for some or all of the finite strings from a given finite or enumerable alphabet we associate to the string a total or partial function from positive integers to positive integers, then there is some total function on positive integers taking only the values $1$ and $2$ that is not associated with any string.
\end{exercise}
\begin{proof}
  Without loss of generality, let $\mathcal A$ be an enumerable alphabet of symbols $A^1, A^2, A^3, \ldots$.
  By an earlier result, the set $S_{\mathcal A}$ of all finite strings is enumerable, so we can list them as follows: $S_1, S_2, \ldots, S_j, \ldots$, where $S_j = A_j^aA_j^bA_j^c\cdots$.
  Each $S_j$ is then associated with a function $f_j$; that is to say, a list of values $f_j(i)$, $i = 1, 2, \ldots$, if such values exist.
  Define then a function $f$ as follows:

  \begin{equation*}
    f(n) = \begin{cases}
      1 &\text{ if } f_n(n) = 2 \\
      2 &\text{ if } f_n(n) \ne 2 \text{ or } f_n(n) \text{ is undefined }
    \end{cases}
  \end{equation*}

  Clearly this function is total on the set of positive integers and is not associated with any finite string: Suppose to the contrary that $f$ is associated with the string $S_m$.
  Then $f(m) = f_m(m)$, absurd by the construction of $f$.
  This completes the proof.
\end{proof}

\begin{exercise}[2.3]
  The set of real numbers (or, equivalently, the set of points on the real number line) is equinumerous with the set of points on the semicircle indicated in Figure 2-3.
\end{exercise}
\begin{proof}
  Draw a line from the center of the semicircle going through the semirciele to the real number line and vary the line's angle.
  As it sweeps the points on the semicircle, it sweeps the points on the real-number line also.
  Clearly this is a one-to-one mapping between the points on the semicircle and the points on the real-number line.
  Further, all points on the semicircle have an image on the real-number line, whence the mapping is surjective and the result follows.
\end{proof}

\begin{exercise}[2.4]
  The set of real numbers $\xi$ with $0 < \xi < 1$ is equinumerous with the set of points on the semicircle.
\end{exercise}
\begin{proof}
  A one-to-one correspondence is given by $(x,y) \mapsto x$.
\end{proof}

\begin{exercise}[2.5]
  The set of real numbers $\xi$ with $0 < \xi < 1$ is equinumerous with the set of all real numbers.
\end{exercise}
\begin{proof}
  By the previous two exercises, the result is obvious.
  However, an analytic correspondence between $(0,1)$ and $(-\pi/2, \pi/2)$ is given by $x \mapsto \pi(x - 1/2)$ and an analytic correspondence between $(-\pi/2, \pi/2)$ and $\mathbb R$ is given by $x \mapsto \tan x$.
  Then $f(x) = \tan \pi(x - 1/2)$ is the desired bijection.
\end{proof}

\begin{exercise}[2.6]
  A real number is called algebraic if and only if it is a solution to some equation of the form
  \begin{equation*}
    c_dx^d + c_{d-1}x^{d-1} + \ldots + c_2x^2 + c_1x + c_0 = 0
  \end{equation*}
  where $c_i$ are rational numbers and $c_d \ne 0$.

  Every algebraic number can be described by a finite string of symbols from an ordinary keyboard.
  Transcendental (non-algebraic) numbers exist.
\end{exercise}
\begin{proof}
  For the first part, clear the denominators of the coefficients by multiplying the whole equation by each of them, this turns the equation into an equivalent one with integer coefficients $a_0, a_1, a_2, \ldots, a_d$.
  Then consider a bijection between the roots of the equations and $d+2$-tuples of the form $(k, a_0, \ldots, a_d)$, where $k$ is the ordinal number of the root of the equation when roots are ordered by their value.
  These tuples are denumerable and the set of finite strings from a finite alphabet is denumerable by an earlier result, whence the set of these tuples is equinumerous with the set of such finite strings; and the result follows.
  Alternatively, assign to each root a finite string taken from the alphabet
  \begin{equation*}
    \text{, 0 1 2 3 4 5 6 7 8 9 -}
  \end{equation*}
  e.g. ``1,2,0,-1'' represents the first root of the equation $2x^2-1 = 0$.

  For the second part, observe that there is an enumerable infinity of algebraic numbers, but the real numbers are nonenumerable, whence there must exist real numbers that are not algebraic.
\end{proof}

\begin{exercise}[2.7]
  The set of real numbers $0 < \xi < 1$ and $\xi$ not a rational number with denominator a power of two is equinumerous with the set of those sets of positive integers that are neither finite nor cofinite.
\end{exercise}
\begin{proof}
  Each such real number $\xi$ has a non-terminating binary expansion $\xi = 0.x_1x_2\ldots$.
  For each such expansion, pick a subset $S$ of the positive integers by letting $n \in S$ if and only if $x_n = 1$.
  Clearly this defines a bijection between the two sets, and the result follows.
\end{proof}

\begin{exercise}[2.8]
  If $A$ is equinumerous with $C$ and $B$ is equinumerous with $D$, and the intersections $A \cap B$ and $C \cap D$ are empty, then the unions $A \cup B$ and $C \cup D$ are equinumerous.
\end{exercise}
\begin{proof}
  Let $f\colon A \to C$ and $g\colon B \to D$ be total bijections.
  Define a function $h\colon A \cup D \to B \cup D$ as follows:
  \begin{equation*}
    h(x) = \begin{cases}
      f(x), &\text{ if }x \in A\\
      g(x), &\text{ if }x \in B\\
    \end{cases}
  \end{equation*}
  Since those two cases are exhaustive and mutually exclusive (the intersections are empty), $h$ is well-defined and is a total bijection.
\end{proof}

\begin{exercise}[2.9]
  The set of real numbers $0 < \xi < 1$ is equinumerous with the set of all sets of positive integers.
\end{exercise}
\begin{proof}
  The result trivially follows from Exercises 1.5, 2.7, and 2.8.
\end{proof}

\begin{exercise}[2.10]
  The following sets are equinumerous: The set $A$ of all pairs of sets of positive integers, the set $B$ of all sets of pairs of positive integers, the set $C$ of all sets of positive integers.
\end{exercise}
\begin{proof}
  By an earlier result, pairs of positive integers are equinumerous with positive integers, so there is a bijection $g$ from positive integers to pairs of positive integers, and its inverse $g^{-1}$.
  Consider then a function $f$ that takes a set $\{i,j,\ldots\}$ of positive integers and associates with it the set $\{g(i), g(j), \ldots\}$ of pairs of positive integers.
  Clearly this function is a total bijection from $C$ to $B$.

  Now consider a map that takes a pair $(X,Y)$ of sets of positive integers to a set of pairs $(x,y)$ for all $x \in X$ and $y \in Y$.
  This map is clearly total, surjective, and injective; this completes the proof.
\end{proof}

\begin{exercise}[2.11]
\end{exercise}

\begin{exercise}[2.12]
\end{exercise}

\begin{exercise}[2.13]
\end{exercise}

\begin{summary}
  \begin{itemize}
    \item This chapter's exercises were mainly a continuation of the previous chapter and its exercises, but here the diagonalization method of proof is introduced.
    If you need to show that a set of objects can never be complete, try it.
    Binary and decimal expansions are very relevant;
    \item $A\cup B$ and $C\cup D$ are equinumerous if $A\cap B = C\cap D = \emptyset$, $A$ is equinumerous with $C$, and $B$ is equinumerous with $D$;
    \item Enumerable sets: Algebraic numbers;
    \item Nonenumerable sets, all equinumerous: Real numbers, any open or closed nondegenerate interval of real numbers, set of points in a plane, set of points in a space, set of all sets of positive integers, set of all pairs of sets of positive integers, set of all sets of pairs of positive integers.
  \end{itemize}
\end{summary}
% section diagonalization (end)
