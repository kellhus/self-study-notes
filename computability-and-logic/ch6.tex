\section{Recursive Functions} % (fold)
\label{sec:recursive_functions}
The zero, successor, and various identity functions (also called projection functions) are together the \emph{basic} functions:

Zero function: $z(0) = 0$, $z(0') = 0$, $z(0'')=0, \ldots$.

Successor function: $s(0) = 0'$, $s(0') = 0''$, $s(0'')=0''', \ldots$.

Identity functions: $\id_i^n(x_1,\ldots,x_i,\ldots,x_n) = x_i$.
\newline
If $f$ is a function of $m$ arguments and each of $g_1,\ldots,g_m$ is a function of $n$ arguments, then the function $h$ obtained by composition (also called substitution) is:
\begin{gather*}
  \boxed{h(x_1, \ldots, x_n) = f(g_1(x_1, \ldots, x_n), \ldots, g_m(x_1, \ldots, x_n))}\\
  \text{Shorthand:}\\
  \boxed{h = \Cn[f,g_1,\ldots,g_m]}
\end{gather*}

We can also obtain new functions by (primitive) recursion:
\begin{gather*}
  \boxed{h(x, 0) = f(x),~~~h(x, s(y)) = g(x, y, h(x, y))}\\
  \text{Shorthand:}\\
  \boxed{h=\Pr[f,g]}
\end{gather*}

Functions obtained from the basic functions by composition and recursion are called primitive recursive.

Minimization is another way to define new functions (which will not be necessarily primitive recursive).
Given a function $f$ of $n + 1$ arguments, minimization yields a total or partial function $h$ of $n$ arguments as follows:
\begin{align*}
  \Mn[f](x_1, \ldots, x_n) = \begin{cases}
    x_{n+1}, &\text{if $f(x_1,\ldots,x_{n+1}) = 0$ and for all $y < x_{n+1}$}\\
    &\text{$f(x_1,\ldots,x_n,y) \ne 0$ and is defined}\\
    \text{undefined} &\text{otherwise}
  \end{cases}
\end{align*}
If $f$ is a total function, then the above definition simplifies to: $\Mn[f](x_1, \ldots, x_n)$ is the smallest $x_{n+1}$ for which $f(x_1,\ldots,x_{n+1}) = 0$, if such an $x_{n+1}$ exists, and undefined otherwise.

The total function $f$ is called regular if for every $x_1,\ldots,x_n$ there is an $x_{n+1}$ such that $f(x_1,\ldots,x_{n+1}) = 0$.
If $f$ is total, then $\Mn[f]$ is total if and only if $f$ is regular.

Functions obtained from the basic functions by $\Cn$, $\Pr$ and $\Mn$ are called the recursive (total or partial) functions.
Recursive functions are effectively computable.
Church's thesis is that effectively computable total functions are all recursive.

\begin{example}
  \newcommand{\rsum}{\operatorname{sum}}
  \newcommand{\rprod}{\operatorname{prod}}
  The sum and product functions are primitive recursive.
  Indeed, we have the following recursion equations for the sum function:
  \begin{gather*}
    \rsum(x,0)=x=\id_1^1(x)\\
    \rsum(x,y')=\rsum(x,y)'=\Cn[s, \id_3^3](x, y, \rsum(x,y))
  \end{gather*}
  Formally, $\rsum=\Pr[\id_1^1,\Cn[s, \id_3^3]]$. And for the product:
  \begin{gather*}
    \rprod(x,0)=z(x)\\
    \rprod(x,y')=x+\rprod(x,y)
  \end{gather*}
  Formally, $\rprod=\Pr[z,\Cn[\rsum,\id_1^3,\id_3^3]]$.
\end{example}

\begin{exercise}
  \begin{equation*}
    \operatorname{fac} = \Cn[\Pr[\operatorname{const}_1, \Cn[\operatorname{prod}, \id_3^3, \Cn[s, \id_2^3]]], \id, \id]
  \end{equation*}
\end{exercise}
\begin{proof}
  \begin{align*}
    \operatorname{fac}(x) &= \Cn[\Pr[\operatorname{const}_1, \Cn[\operatorname{prod}, \id_3^3, \Cn[s, \id_2^3]]], \id, \id](x, x) \\
    &= \Pr[\operatorname{const}_1, \Cn[\operatorname{prod}, \id_3^3, \Cn[s, \id_2^3]]](\id(x), \id(x)) \\
    &= \Pr[\operatorname{const}_1, \Cn[\operatorname{prod}, \id_3^3, \Cn[s, \id_2^3]]](x, x) \\
    &= \begin{cases}
      \operatorname{const}_1(x) = 0! = 1, &\text{ if }x=0\\
      x!\cdot x', &\text{otherwise}
    \end{cases}
  \end{align*}
  Since, denoting $\Pr[\operatorname{const}_1, \Cn[\operatorname{prod}, \id_3^3, \Cn[s, \id_2^3]]]$ as $\operatorname{dummyfac}$,
  \begin{align*}
    &\Cn[\operatorname{prod}, \id_3^3, \Cn[s, \id_2^3]](x, x, \operatorname{dummyfac}(x,x)) =\\
    &\operatorname{prod}(\id_3^3(x,x,\operatorname{dummyfac}(x,x)), \Cn[s, \id_2^3](x,x,\operatorname{dummyfac}(x,x))) =\\
    &\operatorname{dummyfac}(x,x) \cdot s(x)
  \end{align*}
\end{proof}

\begin{exercise}
  Let $f$ be a primitive recursive function.
  Then the functions
  \begin{gather*}
    g(x,y)=f(x,0)+f(x,1)+\ldots+f(x,y)=\sum_{i=0}^y f(x,i)\\
    h(x,y)=f(x,0)\cdot f(x,1)\cdot\ldots\cdot f(x,y)=\prod_{i=0}^y f(x,i)
  \end{gather*}
  are primitive recursive.
\end{exercise}
\begin{proof}
  We have for $g$ the recursion equations
  \begin{gather*}
    g(x,0)=f(x,0)\\
    g(x,y')=g(x,y)+f(x,y')
  \end{gather*}
  Formally,
  \begin{equation*}
    g = \Pr[\Cn[f, \id, z], \Cn[\operatorname{sum}, \id_3^3, \Cn[f, \id_1^3, \Cn[s, \id_2^3]]]]
  \end{equation*}
  Indeed,
  \begin{align*}
    g(x,y) = \begin{cases}
      \Cn[f, \id, z](x) = f(x, z(x)) = f(x, 0) &\text{ if }y = 0\\
      \begin{aligned}
        &\Cn[\operatorname{sum}, \id_3^3, \Cn[f, \id_1^3, \Cn[s, \id_2^3]]](x, y, g(x, y))=\\
        &\operatorname{sum}(g(x, y), \Cn[f, \id_1^3, \Cn[s, \id_2^3]](x, y, g(x, y)))=\\
        &\operatorname{sum}(g(x, y), f(x, \Cn[s,\id_2^3](x, y, g(x, y))))=\\
        &\operatorname{sum}(g(x, y), f(x, s(y)))=\\
        &g(x, y) + f(x, y')
      \end{aligned} &\text{otherwise}
    \end{cases}
  \end{align*}
  The case for $h$ is similar, except $\operatorname{sum}$ is replaced by $\operatorname{prod}$.
\end{proof}

\begin{exercise}
  The exponential or power function $f(x,y) = x \uparrow y$ is primitive recursive.
\end{exercise}
\begin{proof}
  We have the recursion equations:
  \begin{gather*}
    f(x,0) = x \uparrow 0 = 1\\
    f(x,y') = x \cdot (x \uparrow y)
  \end{gather*}
  Or, more formally,
  \begin{equation*}
    \operatorname{exp} = \Pr[\Cn[s, z], \Cn[\operatorname{prod}, \id_1^3, \id_3^3]]
  \end{equation*}
  Indeed,
  \begin{equation*}
    f(x, 0) = \Cn[s, z](x, 0) = s(z) = 1
  \end{equation*}
  \begin{align*}
    f(x, y') &= \Cn[\operatorname{prod}, \id_1^3, \id_3^3](x, y, f(x, y))\\
    &= \operatorname{prod}(x, f(x, y)) \\
    &= x \cdot f(x, y)
  \end{align*}
\end{proof}

\begin{exercise}
  \newcommand{\pred}{\operatorname{pred}}
  Define $\pred(x)$ to be the predecessor $x-1$ of $x$ for $x > 0$, and let $\pred(0) = 0$ by convention.
  Then $\pred$ is primitive recursive.
  \begin{proof}
    Define $\pred(x) = f(x, x)$.
    We have the recursion equations:
    \begin{gather*}
      f(x, 0) = 0\\
      f(x, y') = y
    \end{gather*}
    Formally, $\pred = f = \Pr[z, \id_2^3]$.
    Indeed, $\pred(0) = f(0, 0) = z(0) = 0$ and $\pred(x') = f(x', x') = \id_2^3(x', x, f(x, y)) = x$.
  \end{proof}
\end{exercise}

\begin{exercise}
  \newcommand{\pred}{\operatorname{pred}}
  Define the modified difference function $m(x, y) = x - y$ if $x \ge y$, and let $m(x, y) = 0$ by convention otherwise.
  Then the function $m$ is primitive recursive.
  \begin{proof}
    We have the recursion equations:
    \begin{gather*}
      m(x, 0) = x\\
      m(x, y') = \pred(m(x,y))
    \end{gather*}
    If $x < y$, $\pred$ will reach zero, and the convention will be satisfied.
    Formally, $m = \Pr[\id, \Cn[\pred, \id_3^3]]$
  \end{proof}
\end{exercise}

\begin{exercise}
  \newcommand{\sg}{\operatorname{sg}}
  \newcommand{\sgb}{\overline{\operatorname{sg}}}
  Define $\sg(0) = 0$, and $\sg(x) = 1$ if $x > 0$; also define $\sgb(0) = 1$ and $\sgb(x)  = 0$ if $x > 0$.
  Then $\sg$ and $\sgb$ are primitive recursive.
  \begin{proof}
    We can use the function $m$ from the previous exercise and put $\sg(x) = m(1, m(1, x))$.
    Indeed,
    \begin{align*}
      &\sg(0) = m(1, m(1, 0)) = m(1, 1) = 0\\
      &\sg(1) = m(1, m(1, 1)) = m(1, 0) = 1\\
      &\sg(2) = m(1, m(1, 2)) = m(1, 0) = 1\\
      &\ldots
    \end{align*}
    For the second function, put $\sgb(x) = m(1, x)$:
    \begin{align*}
      &\sg(0) = m(1, 0) = 1\\
      &\sg(x) = m(1, 1) = 0\\
      &\sg(x) = m(1, 2) = 0\\
      &\ldots
    \end{align*}
  \end{proof}
\end{exercise}

\begin{exercise}[6.1]
  Let $f$ be a two-argument recursive total function.
  Then the following functions are also recursive: $g(x,y)=f(y,x)$, $h(x)=f(x,x)$, $k_{17}(x)=f(17,x)$ and $k^{17}(x)=f(x,17)$.
\end{exercise}
\begin{proof}
  For the first function, put $g = \Cn[f, \id_2^2, \id_1^2]$.
  For the second function, put $h = \Cn[f, \id, \id]$.
  For the third function, put $k_{17} = \Cn[f, \operatorname{const}_{17}, \id]$.
  The fourth function is then $k^{17} = \Cn[f, \id, \operatorname{const}_{17}]$.
\end{proof}

\begin{exercise}[6.2]
  Let $J_0(a,b) = (a^2+2ab+b^2-b-3a+2)/2$ be the function coding pairs of positive integers by positive integers, and $J(a,b) = J_0(a+1,b+1)-1$ be the function coding pairs of natural numbers by natural numbers.
  Then $J$ is primitive recursive.
\end{exercise}
\begin{proof}
  \newcommand{\pred}{\operatorname{pred}}
  \newcommand{\sgb}{\overline{\operatorname{sg}}}
  \newcommand{\even}{\operatorname{even}}
  \newcommand{\divtwo}{\operatorname{divtwo}}
  \newcommand{\rsum}{\operatorname{sum}}
  First we need to show that $J_0$ is primitive recursive.
  We have $a^2+2ab+b^2 = b^2+(a+2a)a \ge b+3a$ for $a$ and $b$ positive integers, thus we can compose the modified difference function $m$, the power function, the sum function, and the product function to see that the numerator is primitive recursive.

  For the division part, we only care about even integers, so let's construct a primitive recursive function that is equivalent to $f(x)=x/2$ for even integers only.
  First define an auxiliary function $\even(x) = \sgb(h(x,x))$, where the recursion equations for $h$ are:
  \begin{gather*}
    h(x,0) = 0\\
    h(x,y') = m(x,h(x,y))
  \end{gather*}
  Formally, $h = \Pr[z, \Cn[m,\id^3_1,\id^3_3]]$.
  The $\even$ function so defined is primitive recursive and returns $1$ whenever $x$ is even, and $0$ otherwise.
  Now define $\divtwo(x)=u(x,x)$.
  We have the recursion equations:
  \begin{gather*}
    u(x,0) = 0 \\
    u(x,y') = u(x,y)+\even(y)
  \end{gather*}
  This is the desired function.
  Formally,
  \begin{equation*}
    \divtwo = \Cn[\Pr[z, \Cn[\rsum, \id^3_3, \Cn[\even, \id^3_2]]], \id, \id]
  \end{equation*}
  Indeed,
  \begin{align*}
    &\divtwo(2)=1\\
    &\divtwo(4)=2\\
    &\divtwo(6)=3\\
    &\ldots
  \end{align*}
  Thus we have shown that $J_0$ is primitive recursive.
  The function $J$ is obtained from $J_0$ by composition of known primitive recursive functions:
  \begin{gather*}
    c_1 = \Cn[\operatorname{const}_1, \id^2_1]\\
    J(a,b) = \Cn[m, \Cn[J_0, \Cn[\rsum, \id^2_1, c_1], \Cn[\rsum, \id^2_2, c_1]], c_1](a,b)
  \end{gather*}
  and is thus primitive recursive; this completes the proof.
\end{proof}

\begin{exercise}[6.3]
  The following functions are primitive recursive: the absolute difference $|x-y|$ (defined to be $x-y$ if $y<x$, and $y-x$ otherwise), the order characteristic $\chi_\le(x,y)$ (defined to be $1$ if $x\le y$ and $0$ otherwise), and the maximum $\max(x,y)$.
\end{exercise}
\begin{proof}
  \newcommand{\rsum}{\operatorname{sum}}
  \newcommand{\sg}{\operatorname{sg}}
  For the absolute difference, note that $|x-y|=m(x,y)+m(y,x)$.
  Or, more formally, $\Cn[\rsum, \Cn[m, \id^2_1, \id^2_2], \Cn[m, \id^2_2, \id^2_1]]$.
  For the order characteristic, put $\chi_\le(x,y) = \sg(m(x+1,y))$.
  For the maximum function, put $\max(x,y) = m(x,y)+y$:
  \begin{enumerate}
    \item $x<y \implies m(x,y)+y=x-y+y=x$
    \item $x>y \implies m(x,y)+y=0+y=y$
    \item $x=y \implies m(x,y)+y=0+y=y=x$
  \end{enumerate}
  Less elegantly, $\max(x,y) = \sg(m(x+1,y)) \cdot x + \sg(m(y,x)) \cdot y$.
\end{proof}

\begin{exercise}[6.4]
  The following functions are primitive recursive: $c(x,y,z)=1$ if $yz=x$ and $0$ otherwise, $d(x,y,z)=1$ if $J(y,z)=x$ and $0$ otherwise.
\end{exercise}
\begin{proof}
  \newcommand{\sgb}{\overline{\operatorname{sg}}}
  Put $c(x,y,z) = \sgb(|y\cdot z - x|)$ and $d(x,y,z) = \sgb(|J(y,z) - x|)$.
\end{proof}

\begin{exercise}[6.5]
  Define $K(n)$ and $L(n)$ to be the first and second entries of the pair of natural numbers coded by the number $n$ (under the coding $J$ from an earlier exercise), so that $J(K(n), L(n)) = n$.
  The functions $K$ and $L$ are primitive recursive.
\end{exercise}
\begin{proof}
  \newcommand{\sgb}{\overline{\operatorname{sg}}}
  Observe that $J(x,y)\ge x$ and $J(x,y)\ge y$.
  In other words, $n \ge K(n)$ and $n \ge L(n)$, which establishes useful bounds.
  Further, with some fussing and similarly to an earlier exercise, we know that the functions
  \begin{gather*}
    g(n,x,y)=f(n,x,0)+f(n,x,1)+\ldots+f(n,x,y)=\sum_{i=0}^y f(n,x,i)\\
    h(n,x,y)=g(n,0,y)+g(n,1,y)+\ldots+g(n,x,y)=\sum_{i=0}^x g(n,i,y)
  \end{gather*}
  are primitive recursive when $f$ is primitive recursive.

  Now, we need a three-argument primitive recursive function that accepts $n$, $x$, $y$ and returns $0$ if $(x,y)$ isn't coded by $n$, and $x$ otherwise.
  Similarly to an earlier exercise, we have just such a function: $f(n,x,y) = x \cdot \sgb(|J(x,y) - n|)$.
  But plugging it into the first series gives as a function $h$ such that, applied to the arguments $h(n,n,n)$, it returns an $x$ that is the first entry of the pair coded by $n$.
  Define now $K(n)=h(n,n,n)$, and we are done.

  The argument for $L(n)$ is similar.
\end{proof}

\begin{exercise}[6.6]
  Let $n=2^{k(n)}(2l(n)+1)\mathop{\dot -}1$ be a coding of pairs of natural numbers by natural numbers.
  Then the functions $k$ and $l$ are primitive recursive.
\end{exercise}
\begin{proof}
  The method used in the previous exercise works with trivial modifications.
\end{proof}

\begin{exercise}[6.7]
  Solution: Every recursive function can be represented as a finite string using an enumerable alphabet that is the union of the following symbols:
  \begin{equation*}
    \Cn~~\Pr~~\Mn~~s~~z~~\id~~\_~~\hat{}~~[~~]
  \end{equation*}
  and numerals denoting natural numbers.
  We know from an earlier result that such a set of finite strings is enumerable, and we can assign code numbers to them using the usual methods, e.g. prime decomposition.
\end{exercise}

\begin{exercise}[6.8]
  Let $d(x)=1$ if the one-argument recursive function $f_x$ with code number $x$ is defined and $f_x(x)=0$, and $d(x)=0$ otherwise.
  Then $d$ is not recursive.
\end{exercise}
\begin{proof}
  Suppose towards a contradiction that $d$ is recursive.
  Then it is in the list of enumerated recursive functions for some code number $n$: $d = f_n$.
  Suppose $f_n(n)$ is defined and is zero, then by the construction $d(n)=1$, absurd.
\end{proof}

\begin{exercise}[6.9]
  Let
  \begin{equation*}
    h(x,y) = \begin{cases}
      1, &\text{if recursive $f_x(y)$ is defined}\\
      0, &\text{otherwise}
    \end{cases}
  \end{equation*}
  Then $h$ is not recursive.
\end{exercise}
\begin{proof}
  \newcommand{\const}{\operatorname{const}}
  There exists a recursive function $f_n$ such that $f(0)=0$ and undefined for nonzero arguments.
  (Then $h(n,0)=1$ and $h(n,y)=0$ for $y\ne 0$.)
  
  Suppose towards a contradiction that $h$ is recursive.
  Now construct a one-argument recursive function $f_m$ from it by putting $f_m(y) = f_n(h(m, y))$.
  Now suppose $h(m,0)=0$, then $f_m(0)=0$ and $h(m,0)=1$, absurd; and suppose otherwise $h(m,0)=1$, then $f_m(0)$ is undefined and $h(m,0)=0$, absurd again. This completes the proof.
\end{proof}
% section recursive_functions (end)
