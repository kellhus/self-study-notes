\documentclass{article}

\usepackage{amssymb}
\usepackage{amsthm}
\usepackage{amsmath}
\usepackage{calc}

\theoremstyle{definition}
\newtheorem{exercise}{Exercise}

\theoremstyle{remark}
\newtheorem*{remark}{Remark}
\newtheorem*{proofidea}{Proof idea}

\newcommand\vartextvisiblespace[1][.5em]{%
  \makebox[#1]{%
    \kern.07em
    \vrule height.3ex
    \hrulefill
    \vrule height.3ex
    \kern.07em
  }%
}


\begin{document}

A function that enumerates a set $S$ is a mapping $f$ from a subset of the positive integers $P$ to $S$.
The requirement is that this function be surjective (each element of the set be associated with \emph{at least one} positive integer, and so show up \emph{at least once} in the list), not that the domain be the whole of $P$.
When the domain isn't $P$, we say that $f$ is a \emph{partial function} of positive integers, as opposed to a \emph{total function}.

Ordered pairs of positive integers can be enumerated in two ways:
\begin{enumerate}
  \item Cantor's: pairs come in blocks of increasing sum of entries and within each block pairs appear in order of increasing first entry: (1, 1); (1, 2), (2, 1); (1, 3), (2, 2), (3, 1); (1, 4), (2, 3), (3, 2), (4, 1); \ldots;
  \item Imagine an enumerable infinity of rooms.
  For pairs $(1, n)$, use every other room.
  For pairs $(2, n)$, use every other room \emph{among those remaining vacant}, for pairs $(3, n)$, use every other room then remaining, and so on:
  \begin{itemize}
    \newcommand\room{\vartextvisiblespace[\widthof{$(1,1)$}]}
    \item $(1,1), \room, (1,2), \room, (1,3), \room, (1,4), \room, (1,5), \room, \ldots$
    \item $(1,1), (2,1), (1,2), \room, (1,3), (2,2), (1,4), \room, (1,5), (2,3), \ldots$
    \item $(1,1), (2,1), (1,2), (3,1), (1,3), (2,2), (1,4), \room, (1,5), (2,3), \ldots$
    \item $(1,1), (2,1), (1,2), (3,1), (1,3), (2,2), (1,4), (4,1), (1,5), (2,3), \ldots$
  \end{itemize}
\end{enumerate}

Applying an enumerating function $f$ to a positive integer may be called \emph{decoding}, while going the other way may be called \emph{encoding}.



\begin{exercise}[Boolos 1.1]
  Exercise statement.
\end{exercise}

\begin{proofidea}
  Proof idea.
\end{proofidea}

\begin{proof}
  Proof.
\end{proof}


\end{document}
