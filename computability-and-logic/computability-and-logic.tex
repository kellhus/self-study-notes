\documentclass[a4paper]{article}

\usepackage[left=4.4cm, right=4.4cm, top=1cm, bottom=2cm]{geometry}
\usepackage{amssymb}
\usepackage{amsthm}
\usepackage{amsmath}
\usepackage{calc}
\usepackage{hyperref}
\usepackage{color}

\theoremstyle{definition}
\newtheorem{exercise}{Exercise}
\newtheorem{example}{Example}
\newtheorem{theorem}{Theorem}
\newtheorem{lemma}{Lemma}

\theoremstyle{remark}
\newtheorem*{remark}{Remark}
\newtheorem*{proofidea}{Proof idea}

\newenvironment{summary}
{\centering\textbf{Summary}\footnotesize\setlength\itemsep{1em}}

\newcommand\vartextvisiblespace[1][.5em]{%
  \makebox[#1]{%
    \kern.07em
    \vrule height.3ex
    \hrulefill
    \vrule height.3ex
    \kern.07em
  }%
}
\newcommand\id{\operatorname{id}}
\newcommand\myline{\noindent\makebox[\linewidth]{\rule{\textwidth}{0.4pt}}\break}
\newcommand\tk{{\color{red}\textbf{TK$\implies$}}}


\begin{document}
Note: The textbook is extremely unrigorous and proceeds mainly by way of handwavey, intuitive arguments.
The style of the ``solutions'' is the same.
Garbage in, garbage out.

\section{Enumerability} % (fold)
\label{sec:enumerability}
A function that enumerates a set $S$ is a mapping $f$ from a subset of the positive integers $P$ to $S$.
The requirement is that this function be surjective (each element of the set be associated with \emph{at least one} positive integer, and so show up \emph{at least once} in the list), not that the domain be the whole of $P$.
When the domain isn't $P$, we say that $f$ is a \emph{partial function} of positive integers, as opposed to a \emph{total function}.

Ordered tuples of positive integers can be enumerated in two ways:
\begin{enumerate}
  \item Cantor's: tuples come in blocks of increasing sum of entries and within each block tuples appear in order of increasing first entry: (1, 1); (1, 2), (2, 1); (1, 3), (2, 2), (3, 1); (1, 4), (2, 3), (3, 2), (4, 1); \ldots.
  This defines a function $G\colon P \to P \times P$;
  \item Imagine an enumerable infinity of rooms.
  For tuples $(1, n)$, use every other room.
  For tuples $(2, n)$, use every other room \emph{among those remaining vacant}, for tuples $(3, n)$, use every other room then remaining, and so on:
  \begin{itemize}
    \newcommand\room{\vartextvisiblespace[\widthof{$(1,1)$}]}
    \item $(1,1), \room, (1,2), \room, (1,3), \room, (1,4), \room, (1,5), \room, \ldots$
    \item $(1,1), (2,1), (1,2), \room, (1,3), (2,2), (1,4), \room, (1,5), (2,3), \ldots$
    \item $(1,1), (2,1), (1,2), (3,1), (1,3), (2,2), (1,4), \room, (1,5), (2,3), \ldots$
    \item $(1,1), (2,1), (1,2), (3,1), (1,3), (2,2), (1,4), (4,1), (1,5), (2,3), \ldots$
  \end{itemize}
\end{enumerate}

Applying an enumerating function $f$ to a positive integer may be called \emph{decoding}, while going the other way may be called \emph{encoding}.
For example, the function $J\colon P \times P \to P$, $(m,n) \mapsto (m^2+2mn+n^2-n-3n+2)/2$ is the encoding function for the decoding function $G$.

Enumerating all triples of positive integers can be achieved by building on Cantor's enumeration of all tuples of positive integers.
Define a map $H\colon P \times P \to P \times P \times P$ by $(i,j) \mapsto (i, G(j))$, whence we have:
\begin{align*}
  n &\mapsto G(n) \equiv (K(n), L(n)) \\
    &\mapsto (K(n), G(L(n))) \equiv (K(n), K(L(n)), L(L(n)))
\end{align*}
It is easily seen that no triple is missed.
Indeed, let $(p,q,r)$ be arbitrary; then there is an $m = J(q,r)$ such that $G(m) = (q,r)$, and then there is an $n = J(p,m)$ such that $G(n) = (p,m)$.
The triple associated with $n$ is then precisely $(p,q,r)$, as desired.

This method can be used to enumerate $k$-tuples for any fixed $k$ by replacing the last component $n$ in tuples with the $k-1$-tuple that appears in the $n$th place in the enumeration of all $k-1$-tuples.

In turn, the enumeration of all $k$-tuples allows us to enumerate all finite sequences by tuples of positive integers (and thus, indirectly, by single numbers).
Let $G_1, G_2 \equiv G, G_3, G_4, \ldots$ be $k$-tuple enumerations.
Then tuples $(k, a)$ code for sequences $s$ of length $k$ where $G_k(a) = s$.
Another approach is possible using prime decomposition: $2^i3^j5^k7^m11^n\ldots \mapsto (i,j,k,m,n,\ldots) = s$, owing to the fact every integer greater than $1$ can be decomposed in a unique way.

\begin{exercise}[1.1]
  Let $f\colon A \to B$ be injective and $f\inv$ be its inverse (\emph{C\&L} defines the inverse in a weird way, so this is legitimate \ldots).
  Then $f\inv$ is total if and only if $f$ is surjective, and $f\inv$ is surjective if and only if $f$ is total.
\end{exercise}
\begin{proof}
  Let $f$ be surjective.
  Then for every $b \in B$ there is some $a \in A$ such that $f(a) = b$.
  Moreover, this $a$ is unique by the injectivity of $f$.
  This defines a total inverse.
  Now let $f\inv$ be total.
  Then it is defined for every element $b \in B$, whence for every $b \in B$ there is one and only $a \in A$ such that $f(a) = b$, and the surjectivity of $f$ follows.

  To show the converse, note that $f^{-1-1} = f$.
  Let then $f^{-1-1} = f$ be total, whence $f\inv$ is surjective by the first assertion, and let $f\inv$ be surjective, whence $f^{-1-1} = f$ is again total, as desired.
\end{proof}

\begin{exercise}[1.2]
  Let $f\colon A \to B$ and $g\colon B\to C$.
  Define $h = gf \colon A \to C$ by $h(a) = g(f(a))$.
  \begin{itemize}
    \item Let $f$ and $g$ be both total.
    $f(a)$ is defined for every $a \in A$, and $g(f(a))$ is defined for every $f(a) \in B$, whence $gf$ is total.
    \item Let $f$ and $g$ be both surjective.
    By the surjectivity of $g$, for each $c \in C$ there is a $b \in B$ such that $g(b) = c$.
    By the surjectivity of $f$, for each $b \in B$ there is an $a \in A$ such that $f(a) = b$.
    Combining these, we see that for every $c \in C$ there is an $a \in A$ such that $gf(a) = c$, whence $gf$ is surjective also.
    \item Let $f$ and $g$ be both injective.
    Then $gf$ is injective also:
    \begin{align*}
      g(f(a')) = g(f(a'')) \implies f(a') = f(a'') \implies a' = a''
    \end{align*}
  \end{itemize}
\end{exercise}

\begin{exercise}[1.3]
  A correspondence between $A$ and $B$ is a total, surjective and injective function from $A$ to $B$.
  Two sets are equinumerous if and only if there is a correspondence between them.
  \begin{itemize}
    \item Any set $A$ is equinumerous with itself.
    Indeed, take the identity function $\id\colon A \to A$ defined by $a \mapsto a$.
    Clearly, $\id$ is injective ($\id(a') = \id(a'') \implies a' = a''$), surjective (for every $a \in A$ there is an $a = a = \id(a) \in A$), and total.
    \item If $A$ is equinumerous with $B$, then $B$ is equinumerous with $A$.
    Let $f$ be a total bijection from $A$ to $B$, and let $f\inv$ be its inverse.
    By an earlier result, $f\inv$ is total and surjective.
    The injectivity of $f\inv$ follows from the definition, whence $f\inv$ is a correspondence between $B$ and $A$, and the result follows.
    \item If $A$ is equinumerous with $B$ and $B$ is equinumerous with $C$, then $A$ is equinumerous with $C$.
    By the previous exercise, $gf = g \circ f$ is a correspondence, as desired.
  \end{itemize}
\end{exercise}

\begin{exercise}[1.4]
  A set $A$ has $n$ elements if and only if it is equinumerous with $\{1,2,\ldots,n\}$.
  A nonempty set $A$ is finite if and only if there exists a positive integer $n$ such that $A$ has $n$ elements.
  Any enumerable set is either finite or equinumerous with the set of all positive integers.
\end{exercise}
\begin{proof}
  Without loss of generality, let $A$ be an infinite subset of the positive integers $P$.
  Define a sequence of infinite subsets
  \begin{equation*}
    \ldots \subset A_n \subset \ldots \subset A_2 \subset A_1
  \end{equation*}
  as follows: put $A_1 = A$.
  By the well-ordering of $P$, the subset $A_n$ has a least element $a_n$.
  Define $f(n) = a_n$ and put $A_{n+1} = A_n \setminus \{a_n\}$.
  Since $A_n$ was infinite, it follows that $A_{n+1}$ is also infinite.
  To show that the $f$ so defined is injective, suppose $f(A') = f(A'')$.
  Then $A'$ and $A''$ have the same least element $a$ and are the same by the construction.
  To show that $f$ is surjective, let $a \in A$ be arbitrary.
  Then there exists an $A_n$ such that $a$ is its least element, whence $f(n) = a$ is defined.
  We have a bijection from $P$ to $A$, which shows that $A$ is equinumerous with $P$, as desired.
\end{proof}

\begin{exercise}[1.5]
  Let $X$ be the set of rational numbers $\pm\frac{m}{n}$ (written in lowest terms) where $n = 2^k$ is some power of two.
  Let $Y$ be the set of those subsets of $P$ that are either finite or cofinite (a subset of $P$ is cofinite if its complement in $P$ is finite).
  These sets are equinumerous.
\end{exercise}
\begin{proof}
  Each positive element in $X$ can be uniquely represented as a tuple $(m,k)$ where $m$ is the numerator and $k$ is the power of the denominator, whence $X$ is equinumerous with the rational numbers; and since we know the rational numbers are denumerable, $X$ is denumerable also.

  Now enumerate $Y$ as follows: from Example 1.10, we know that finite subsets of the positive integers are enumerable.
  List them as $Y_1,Y_2, \ldots$ and put their respective complements $Y_{-1}, Y_{-2}$ in between: $Y_1, Y_{-1}, Y_2, Y_{-2}, \ldots$.
  Clearly, every finite or cofinite subset occurs in this sequence, whence $Y$ is denumerable.

  Since $X$ and $Y$ are both denumerable, they are both equinumerous with the positive integers by an earlier exercise, and thus equinumerous with each other.
\end{proof}

\begin{exercise}[1.6]
  The set of all finite subsets of an enumerable set $A$ is enumerable.
\end{exercise}
\begin{proof}
  Since $A$ is enumerable, its elements can be listed as the sequence
  \begin{equation*}
    a_1, a_2, a_3, a_4, \ldots
  \end{equation*}
  We can enumerate the finite subsets of $A$ as follows: first write down the empty set, then write down the subsets such that the indices of their elements $a_1, \ldots, a_k$ add up to $1$ ($1 + \ldots + k = 1$), then the subsets with indices adding up to $2$, and so on:
  \begin{equation}
    \emptyset, \{a_1\}, \{a_2\}, \{a_1, a_2\}, \{a_3\}, \ldots
  \end{equation}
  Clearly, each finite set $A' \subset A$ occurs in this sequence: it consists of finitely many elements $a_k$ whose finite indices add up to some finite $n$.
\end{proof}

\begin{exercise}[1.7]
  Let $A = \{A_1, A_2, \ldots\}$ be an enumerable family of enumerable sets $A_i$.
  The union $\cup A$ of the family $A$ is enumerable.
\end{exercise}
\begin{proof}
  Using the axiom of choice, pick an enumeration $a_{i1}, a_{i2}, a_{i3}, a_{i4}, \ldots$ for each $A_i$.
  Clearly, the members of $\cup A$ are elements $a_{ij}$ for some finite $i,j \in P$, whence $\cup A$ is enumerable by Cantor's method of enumerating all pairs of positive integers.
\end{proof}

\begin{summary}
  \begin{itemize}
    \item A subset of positive integers contains a smallest element (well-ordering principle);
    \item To show that a set is enumerable, list its elements in a sequence that doesn't get ``stuck'' when listing an infinite block of elements.
    Repetitions aren't important, just make sure each element of interest occurs at least once in the sequence.
    Alternatively, demonstrate a correspondence between the set and another set known to be enumerable;
    \item Examples: positive integers, the set of $k$-tuples of positive integers for any fixed $k$, rational numbers, the set of finite sequences of positive integers, the set of finite subsets of an enumerable set, any subset of an enumerable set, and the union of an enumerable family of enumerable sets;
    \item Equinumerosity is reflexive, transitive, and symmetric;
    \item An enumerable set is either finite or equinumerous with the positive integers.
  \end{itemize}
\end{summary}
% section enumerability (end)

\section{Diagonalization} % (fold)
\label{sec:diagonalization}
\begin{theorem}[Cantor]
  The set of all sets of positive integers is not enumerable.
\end{theorem}
\begin{proof}
  Let $L = S_1, S_2, S_3, \ldots$ be an arbitrary list of sets of positive integers.
  Let $S'$ be a set of positive integers such that for each $n$, $n \in S'$ if and only if $n \not\in S_n$.

  Suppose there exists some $m$ such that $S' = S_m$, that is, $S'$ is on the list $L$ for some $m$.
  Then, by the construction of $S'$, $m \in S_m$ if and only if $m \not\in S_m$, absurd; whence there exists no such $m$, and $S'$ is not on the list.

  Since $L$ was arbitrary, this method works for any list $L$, in particular, the list $L$ of all sets of positive integers.
  Then no list enumerates the power set of positive integers; this completes the proof.
\end{proof}

\begin{theorem}
  The set of real numbers is not enumerable.
\end{theorem}
\begin{proof}
  Suppose there exists an enumeration $f\colon \mathbb N \to \mathbb R$.
  Then for each $x \in \mathbb R$ there exists an $n \in \mathbb N$ such that $f(n) = x$, as $f$ is surjective.
  Let $d_{n,0}$ be the digit before the decimal point of $f(n)$, and $d_{n,i}$ the $i$th digit of the decimal expansion of $f(n)$.
  For all $i$, let $e_i$ be a digit different from $d_{i,i}$.
  Now construct a real number $r$ whose decimal expansion is $e_0.e_1e_2\ldots$.
  For each $n \in \mathbb N$, we have $e_n \ne d_{n,n}$, and so $r \ne f(n)$.
  Then $r$ is a real number not in the image of $f$, whence $f$ isn't surjective.
\end{proof}

\begin{exercise}[2.1]
  The set of all subsets of an infinite enumerable set $A$ is nonenumerable.
\end{exercise}
\begin{proof}
  By an earlier exercise, there exists a correspondence $f\colon P \to A$.
  Then there is a correspondence $h\colon \mathcal P(A) \to \mathcal P(P)$ given by $\{a_i, a_j, \ldots\} \mapsto \{i, j, \ldots\}$.
  But by Cantor's theorem the power set of positive integers is nonenumerable, whence the power set of $A$ is nonenumerable also.
\end{proof}
\begin{proof}
  Suppose the power set of $A$ is enumerable.
  Then there exists a list $A_1,A_2,\ldots$ of subsets of $A$.
  Since each subset of an enumerable set is enumerable, we can write each subset $A_j$ of $A$ as a list $a_{j,1},a_{j,2},a_{j,3},\ldots$.
  Let $B$ be a subset such that for each $i$, $b_i \in B$ is different from $a_{i,i} \in A_i$.
  By the assumption, $B = A_n$ for some $n$.
  Examine the element $b_n$.
  By the construction, $b_n \ne a_{n,n} = b_n$, absurd; and it follows that the power set of $A$ is nonenumerable.
\end{proof}

\begin{exercise}[2.2]
  If for some or all of the finite strings from a given finite or enumerable alphabet we associate to the string a total or partial function from positive integers to positive integers, then there is some total function on positive integers taking only the values $1$ and $2$ that is not associated with any string.
\end{exercise}
\begin{proof}
  Without loss of generality, let $\mathcal A$ be an enumerable alphabet of symbols $A^1, A^2, A^3, \ldots$.
  By an earlier result, the set $S_{\mathcal A}$ of all finite strings is enumerable, so we can list them as follows: $S_1, S_2, \ldots, S_j, \ldots$, where $S_j = A_j^aA_j^bA_j^c\cdots$.
  Each $S_j$ is then associated with a function $f_j$; that is to say, a list of values $f_j(i)$, $i = 1, 2, \ldots$, if such values exist.
  Define then a function $f$ as follows:

  \begin{equation*}
    f(n) = \begin{cases}
      1 &\text{ if } f_n(n) = 2 \\
      2 &\text{ if } f_n(n) \ne 2 \text{ or } f_n(n) \text{ is undefined }
    \end{cases}
  \end{equation*}

  Clearly this function is total on the set of positive integers and is not associated with any finite string: Suppose to the contrary that $f$ is associated with the string $S_m$.
  Then $f(m) = f_m(m)$, absurd by the construction of $f$.
  This completes the proof.
\end{proof}

\begin{exercise}[2.3]
  The set of real numbers (or, equivalently, the set of points on the real number line) is equinumerous with the set of points on the semicircle indicated in Figure 2-3.
\end{exercise}
\begin{proof}
  Draw a line from the center of the semicircle going through the semirciele to the real number line and vary the line's angle.
  As it sweeps the points on the semicircle, it sweeps the points on the real-number line also.
  Clearly this is a one-to-one mapping between the points on the semicircle and the points on the real-number line.
  Further, all points on the semicircle have an image on the real-number line, whence the mapping is surjective and the result follows.
\end{proof}

\begin{exercise}[2.4]
  The set of real numbers $\xi$ with $0 < \xi < 1$ is equinumerous with the set of points on the semicircle.
\end{exercise}
\begin{proof}
  A one-to-one correspondence is given by $(x,y) \mapsto x$.
\end{proof}

\begin{exercise}[2.5]
  The set of real numbers $\xi$ with $0 < \xi < 1$ is equinumerous with the set of all real numbers.
\end{exercise}
\begin{proof}
  By the previous two exercises, the result is obvious.
  However, an analytic correspondence between $(0,1)$ and $(-\pi/2, \pi/2)$ is given by $x \mapsto \pi(x - 1/2)$ and an analytic correspondence between $(-\pi/2, \pi/2)$ and $\mathbb R$ is given by $x \mapsto \tan x$.
  Then $f(x) = \tan \pi(x - 1/2)$ is the desired bijection.
\end{proof}

\begin{exercise}[2.6]
  A real number is called algebraic if and only if it is a solution to some equation of the form
  \begin{equation*}
    c_dx^d + c_{d-1}x^{d-1} + \ldots + c_2x^2 + c_1x + c_0 = 0
  \end{equation*}
  where $c_i$ are rational numbers and $c_d \ne 0$.

  Every algebraic number can be described by a finite string of symbols from an ordinary keyboard.
  Transcendental (non-algebraic) numbers exist.
\end{exercise}
\begin{proof}
  For the first part, clear the denominators of the coefficients by multiplying the whole equation by each of them, this turns the equation into an equivalent one with integer coefficients $a_0, a_1, a_2, \ldots, a_d$.
  Then consider a bijection between the roots of the equations and $d+2$-tuples of the form $(k, a_0, \ldots, a_d)$, where $k$ is the ordinal number of the root of the equation when roots are ordered by their value.
  These tuples are denumerable and the set of finite strings from a finite alphabet is denumerable by an earlier result, whence the set of these tuples is equinumerous with the set of such finite strings; and the result follows.
  Alternatively, assign to each root a finite string taken from the alphabet
  \begin{equation*}
    \text{, 0 1 2 3 4 5 6 7 8 9 -}
  \end{equation*}
  e.g. ``1,2,0,-1'' represents the first root of the equation $2x^2-1 = 0$.

  For the second part, observe that there is an enumerable infinity of algebraic numbers, but the real numbers are nonenumerable, whence there must exist real numbers that are not algebraic.
\end{proof}

\begin{exercise}[2.7]
  The set of real numbers $0 < \xi < 1$ and $\xi$ not a rational number with denominator a power of two is equinumerous with the set of those sets of positive integers that are neither finite nor cofinite.
\end{exercise}
\begin{proof}
  Each such real number $\xi$ has a non-terminating binary expansion $\xi = 0.x_1x_2\ldots$.
  For each such expansion, pick a subset $S$ of the positive integers by letting $n \in S$ if and only if $x_n = 1$.
  Clearly this defines a bijection between the two sets, and the result follows.
\end{proof}

\begin{exercise}[2.8]
  If $A$ is equinumerous with $C$ and $B$ is equinumerous with $D$, and the intersections $A \cap B$ and $C \cap D$ are empty, then the unions $A \cup B$ and $C \cup D$ are equinumerous.
\end{exercise}
\begin{proof}
  Let $f\colon A \to C$ and $g\colon B \to D$ be total bijections.
  Define a function $h\colon A \cup D \to B \cup D$ as follows:
  \begin{equation*}
    h(x) = \begin{cases}
      f(x), &\text{ if }x \in A\\
      g(x), &\text{ if }x \in B\\
    \end{cases}
  \end{equation*}
  Since those two cases are exhaustive and mutually exclusive (the intersections are empty), $h$ is well-defined and is a total bijection.
\end{proof}

\begin{exercise}[2.9]
  The set of real numbers $0 < \xi < 1$ is equinumerous with the set of all sets of positive integers.
\end{exercise}
\begin{proof}
  The result trivially follows from Exercises 1.5, 2.7, and 2.8.
\end{proof}

\begin{exercise}[2.10]
  The following sets are equinumerous: The set $A$ of all pairs of sets of positive integers, the set $B$ of all sets of pairs of positive integers, the set $C$ of all sets of positive integers.
\end{exercise}
\begin{proof}
  By an earlier result, pairs of positive integers are equinumerous with positive integers, so there is a bijection $g$ from positive integers to pairs of positive integers, and its inverse $g^{-1}$.
  Consider then a function $f$ that takes a set $\{i,j,\ldots\}$ of positive integers and associates with it the set $\{g(i), g(j), \ldots\}$ of pairs of positive integers.
  Clearly this function is a total bijection from $C$ to $B$.

  Now consider a map that takes a pair $(X,Y)$ of sets of positive integers to a set of pairs $(x,y)$ for all $x \in X$ and $y \in Y$.
  This map is clearly total, surjective, and injective; this completes the proof.
\end{proof}

\begin{exercise}[2.11]
\end{exercise}

\begin{exercise}[2.12]
\end{exercise}

\begin{exercise}[2.13]
\end{exercise}

\begin{summary}
  \begin{itemize}
    \item This chapter's exercises were mainly a continuation of the previous chapter and its exercises, but here the diagonalization method of proof is introduced.
    If you need to show that a set of objects can never be complete, try it.
    Binary and decimal expansions are very relevant;
    \item $A\cup B$ and $C\cup D$ are equinumerous if $A\cap B = C\cap D = \emptyset$, $A$ is equinumerous with $C$, and $B$ is equinumerous with $D$;
    \item Enumerable sets: Algebraic numbers;
    \item Nonenumerable sets, all equinumerous: Real numbers, any open or closed nondegenerate interval of real numbers, set of points in a plane, set of points in a space, set of all sets of positive integers, set of all pairs of sets of positive integers, set of all sets of pairs of positive integers.
  \end{itemize}
\end{summary}
% section diagonalization (end)

\section{Turing Computability} % (fold)
\label{sec:turing_computability}

% section turing_computability (end)

\section{Uncomputability} % (fold)
\label{sec:uncomputability}
By Exercise 9 (2.2), the set of functions from positive integers to positive integers is nonenumerable, while the set of finite strings (in particular, strings representing Turing machines and therefore Turing-computable functions) is enumerable.
Then there must exist functions that aren't Turing-computable.

Take this Turing machine:
\begin{equation*}
  q_1S_0Rq_3, q_1S_1S_0q_2, q_2S_0Rq_1, q_3S_0S_1q_4, q_3S_1S_0q_2
\end{equation*}
To set some machinery up, modify the quadruple representation as follows: For each $S_j$, ensure there is a quadruple beginning with $q_iS_j$:
\begin{equation*}
  q_1S_0Rq_3, q_1S_1S_0q_2, q_2S_0Rq_1, q_2S_1S_1q4, q_3S_0S_1q_4, q_3S_1S_0q_2
\end{equation*}
But this makes the $q_iS_j$s unnecessary, and we can leave the last two symbols (the instruction and the next state), obtaining:
\begin{equation*}
  Rq_3, S_0q_2, Rq_1, S_1q_4, S_1q_4, S_0q_2
\end{equation*}
Now map $q_i \mapsto i$, $S_j \mapsto j+1$ (to avoid zero), $L \mapsto 3$, $R \mapsto 4$ and obtain:
\begin{equation*}
  4, 3, 1, 2, 4, 1, 2, 4, 2, 4, 1, 2
\end{equation*}
Then this Turing machine is represented by a finite sequence of positive integers or even one positive integer using prime decomposition.

\begin{theorem}
  The halting function $h$ is not Turing computable, where $h(m,n) = 1$ if the Turing machine $m$ halts starting with input $n$, and $2$ otherwise.
\end{theorem}
\begin{proofidea}
  Assume a machine $H$ computing $h$ exists, and construct an absurd machine $M$ that halts when it doesn't halt.
\end{proofidea}
\begin{proof}
  First we need two machines: the copying machine $C$ with states $1,\ldots,p$ that duplicates a block of $n$ strokes on an otherwise blank tape, and the dithering machine $D = 1,3,4,2,3,1,3,3$ that halts on a block of $n$ strokes on an otherwise blank tape if $n > 1$, but never halts if $n=1$.

  Now suppose machine $H$ with states $1,\ldots,q$ computes $h$.
  Combine $C$ and $H$ by renumbering the states of $H$ as $p + 1, \ldots, p + q$ and appending those renumbered instructions after the instructions of $C$.
  This means $C$ used to halt at $p+1$, but now $H$ is executed instead; giving us a machine $G$ with states $1,\ldots,r$ (where $r=p+q$) that computes $g(n)=h(n,n)$.

  Now combine $G$ and $D$ into a machine $M$: renumber $D$'s states $r+1,r+2$ and write these instructions after those of $G$.
  Then
  \begin{equation*}
    M(n) \begin{cases}
      \text{halts} &\text{ if }h(n,n)=2\text{, i.e. machine $n$ doesn't halt on $n$}\\
      \text{doesn't halt} &\text{ if }h(n,n)=1\text{, i.e. machine $n$ halts on $n$}
    \end{cases}
  \end{equation*}
  Now run $M$ on its own number $m$.
  We have that $M(m)$ halts if $M(m)$ doesn't halt, absurd.
  This completes the proof.
\end{proof}

Define the score of a $k$-state Turing machine $M_k$ as follows:
\begin{equation*}
  \text{score of $M_k$} = \begin{cases}
    0, &\text{ if $M_k$(k) never halts or halts in a nonstandard position}\\
    n, &\text{ if $M_k(k)$ halts with output $n$}
  \end{cases}
\end{equation*}
Now define the scoring function
\begin{equation*}
  s(k) = \max\limits_k \left(\text{score of $M_k$}\right)
\end{equation*}
That is, $s(k)$ is the highest score achieved by any $k$-state Turing machine $M_k$, for all $k$.

\begin{theorem}
  The scoring function $s$ is not Turing computable.
\end{theorem}
\begin{proof}
  Suppose $s$ is Turing computable (there exists a machine $M_s$ computing it).
  Then so is the function $t$, given by $t(k) = s(k) + 1$.
  Indeed, modify $M_s$ to move left, print a stroke and halt instead of halting.
  But no Turing machine can compute $t$: Let $M_t$ be such a machine and $k$ its number of states.
  Then $M_t(k) = t(k) = s(k) + 1 > s(k)$, absurd.
\end{proof}

Busy beaver: a Turing machine starts with a blank tape.
If the machine halts in a standard configuration scanning an $n$-block of $S_1$s on an otherwise blank tape, its productivity is $n$, otherwise $0$.
Now define
\begin{equation*}
  p(n) = \max \left(\text{productivity of any Turing machine with at most $n$ states}\right)
\end{equation*}
This busy-beaver function is also Turing uncomputable.
The following are facts about $p$: $p(1) = 1$, $p(n+1)>p(n)$ for all $n$, there is an $i$ such that $p(n+i) \ge p(n)$ for all $n$.

\begin{exercise}[4.1]
  Solution: Alternate rightwards and leftwards from the starting point in steps of increasing length, putting a marker if the destination is blank or halting if finding something that is not the marker.
  If the tape is not completely blank, there has to be a $1$ within a finite length of the initial position.
\end{exercise}

\begin{exercise}[4.2]
  Solution: No such Turing machine exists.
  The tape is infinite in both directions, so no matter how many steps we take in some direction, we can still take that many plus one steps; and we \emph{still} won't know if the tape is completely blank or not, so we can never halt.
\end{exercise}

\begin{exercise}[4.3]
  Copying machine $C$ solution:
  \begin{align*}
    q_1S_0Lq_7,&~~q_1S_1S_0q_2&\text{(go parking if blank or put a marker if not)}\\
    q_2S_0Rq_3,&~~q_2S_1Rq_1&\text{(helper routines)}\\
    q_3S_0Rq_4,&~~q_3S_1Rq_3&\text{(get to the end of the copy)}\\
    q_4S_0S_1q_5,&~~q_4S_1Rq_4&\text{(get to the end of the copy and write a $1$)}\\
    q_5S_0Lq_6,&~~q_5S_1Lq_5&\text{(go back to the marker)}\\
    q_6S_0S_1q_2,&~~q_6S_1Lq_6&\text{(go back to the marker and overwrite it with $1$)}\\
    q_7S_0Rq_8,&~~q_7S_1Lq_7&\text{(parking in the standard final position)}
  \end{align*}
  The idea is to put a blank marker at the start of the string of $1$s, then skip the $1$s until a blank occurs, skip the blank (and $1$s, if any) and write a $1$, move back to the marker, replace it with $1$, move right and put a marker again, and so on until the marker and the blank between the copied strings co-occur.
  When that happens, park in the standard final configuration and we are done.
\end{exercise}

\begin{exercise}[4.4]
  If a two-place function $g$ is Turing computable, then so is the function $f$ given by $f(n) = g(n,n)$.
\end{exercise}
\begin{proof}
  The copying machine $C$ has states $1,\ldots,p$
  Now let $G$ be the Turing machine that computes $g$ with states $1,\ldots,q$.
  Renumber the states of $G$ as $p+1,\ldots,p+q$ and append them after the states of $C$.
  This gives us a Turing machine that copies the input $n$ and applies the function $g$ to those inputs, which computes the function $f$ as desired.
\end{proof}

\begin{exercise}[4.5]
  Let $U$ be a Turing machine (computing a two-place function $u$) such that for any other Turing machine $M_n$ (computing a one-place function $m_n$) and any $x$, we have that $u(n,x) = m_n(x)$.
  This Turing machine $U$ is called a universal Turing machine (UTM).

  If Turing's thesis is correct, then a UTM must exist.
\end{exercise}
\begin{proof}
  \tk See \href{http://math.stackexchange.com/questions/1408692/}{http://math.stackexchange.com/questions/1408692/} and correct the ``solution'' when you get to recursive enumerability and fixed points to understand the answer therein properly.

  The set of Turing machines is enumerable, then for any finite $n$ we can proceed to enumerate the machines as finite strings.
  Once we get to the required $n$, we run the Turing machine $M_n$ on the argument $x$.
  If $M_n$ halts in a standard position with some output, then we have found $u(n,x)$.
  If $M_n$ does not halt or halts in a nonstandard position, then \emph{that} is the desired ``value'' ($u(n,x)$ is then undefined).
  This is an informal list of instructions to compute $u(n,x)$, so $u$ is effectively computable, and so by Turing's thesis is also Turing-computable, whence $U$ must exist.
\end{proof}
% section uncomputability (end)

\section{Abacus Computability} % (fold)
\label{sec:abacus_computability}
An abacus machine has registers $R_0, R_1, \ldots$.
Register $n$ contains number $[n]$.

A function $f$ of arguments $x_1,\ldots,x_r$ is abacus-computable given the following: \textbf{(1)} The first $r$ registers store the arguments $x_1 = [1],\ldots,x_r=[r]$ and other registers are empty $0=[r+1]=[r+2]=\ldots$; \textbf{(2)} When the computation halts, $f(x_1,\ldots,x_r)=[n]$; \textbf{(3)} If the computation never halts, $f(x_1,\ldots,x_r)$ is undefined.

Every abacus-computable function is Turing-computable.
A way to convert an abacus machine into an equivalent Turing machine is presented in \emph{C\&l}.

We have three trivial functions:
\begin{itemize}
  \item Zero function $z$, $x \mapsto 0$;
  \item Successor function $s$, $x \mapsto x+1$;
  \item Identity function $\id^m_n$, $x_1,\ldots,x_n \mapsto x_m$ ($1 \le m \le n$).
\end{itemize}

Also we have three processes for defining new functions:
\begin{itemize}
  \item Composition (also called substitution), e.g.:
  \begin{equation*}
    h(x_1,x_2,x_3) = f(g_1(x_1,x_2,x_3), g_2(x_1,x_2,x_3))
  \end{equation*}
  \item (Primitive) recursion, e.g.:
  \begin{gather*}
    h(x,0) = f(x)\\
    h(x,y+1) = g(x,y,h(x,y))
  \end{gather*}
  \item Minimization, e.g.: We have a two-argument function $f$, then we define a one-argument function $h$: If $f(x,0),\ldots,f(x,i-1)$ are all defined and nonzero, and $f(x,i)$ is zero, then $h(x)=i$.
\end{itemize}

Functions obtained from the trivial functions by applying these processes are called recursive.
All recursive functions are abacus-computable, and hence Turing-computable.
% section abacus_computability (end)

\section{Recursive Functions} % (fold)
\label{sec:recursive_functions}

% section recursive_functions (end)


\end{document}
