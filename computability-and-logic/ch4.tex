\section{Uncomputability} % (fold)
\label{sec:uncomputability}
By Exercise 9 (2.2), the set of functions from positive integers to positive integers is nonenumerable, while the set of finite strings (in particular, strings representing Turing machines and therefore Turing-computable functions) is enumerable.
Then there must exist functions that aren't Turing-computable.

Take this Turing machine:
\begin{equation*}
  q_1S_0Rq_3, q_1S_1S_0q_2, q_2S_0Rq_1, q_3S_0S_1q_4, q_3S_1S_0q_2
\end{equation*}
To set some machinery up, modify the quadruple representation as follows: For each $S_j$, ensure there is a quadruple beginning with $q_iS_j$:
\begin{equation*}
  q_1S_0Rq_3, q_1S_1S_0q_2, q_2S_0Rq_1, q_2S_1S_1q4, q_3S_0S_1q_4, q_3S_1S_0q_2
\end{equation*}
But this makes the $q_iS_j$s unnecessary, and we can leave the last two symbols (the instruction and the next state), obtaining:
\begin{equation*}
  Rq_3, S_0q_2, Rq_1, S_1q_4, S_1q_4, S_0q_2
\end{equation*}
Now map $q_i \mapsto i$, $S_j \mapsto j+1$ (to avoid zero), $L \mapsto 3$, $R \mapsto 4$ and obtain:
\begin{equation*}
  4, 3, 1, 2, 4, 1, 2, 4, 2, 4, 1, 2
\end{equation*}
Then this Turing machine is represented by a finite sequence of positive integers or even one positive integer using prime decomposition.

\begin{theorem}
  The halting function $h$ is not Turing computable, where $h(m,n) = 1$ if the Turing machine $m$ halts starting with input $n$, and $2$ otherwise.
\end{theorem}
\begin{proofidea}
  Assume a machine $H$ computing $h$ exists, and construct an absurd machine $M$ that halts when it doesn't halt.
\end{proofidea}
\begin{proof}
  First we need two machines: the copying machine $C$ with states $1,\ldots,p$ that duplicates a block of $n$ strokes on an otherwise blank tape, and the dithering machine $D = 1,3,4,2,3,1,3,3$ that halts on a block of $n$ strokes on an otherwise blank tape if $n > 1$, but never halts if $n=1$.

  Now suppose machine $H$ with states $1,\ldots,q$ computes $h$.
  Combine $C$ and $H$ by renumbering the states of $H$ as $p + 1, \ldots, p + q$ and appending those renumbered instructions after the instructions of $C$.
  This means $C$ used to halt at $p+1$, but now $H$ is executed instead; giving us a machine $G$ with states $1,\ldots,r$ (where $r=p+q$) that computes $g(n)=h(n,n)$.

  Now combine $G$ and $D$ into a machine $M$: renumber $D$'s states $r+1,r+2$ and write these instructions after those of $G$.
  Then
  \begin{equation*}
    M(n) \begin{cases}
      \text{halts} &\text{ if }h(n,n)=2\text{, i.e. machine $n$ doesn't halt on $n$}\\
      \text{doesn't halt} &\text{ if }h(n,n)=1\text{, i.e. machine $n$ halts on $n$}
    \end{cases}
  \end{equation*}
  Now run $M$ on its own number $m$.
  We have that $M(m)$ halts if $M(m)$ doesn't halt, absurd.
  This completes the proof.
\end{proof}

Define the score of a $k$-state Turing machine $M_k$ as follows:
\begin{equation*}
  \text{score of $M_k$} = \begin{cases}
    0, &\text{ if $M_k$(k) never halts or halts in a nonstandard position}\\
    n, &\text{ if $M_k(k)$ halts with output $n$}
  \end{cases}
\end{equation*}
Now define the scoring function
\begin{equation*}
  s(k) = \max\limits_k \left(\text{score of $M_k$}\right)
\end{equation*}
That is, $s(k)$ is the highest score achieved by any $k$-state Turing machine $M_k$, for all $k$.

\begin{theorem}
  The scoring function $s$ is not Turing computable.
\end{theorem}
\begin{proof}
  Suppose $s$ is Turing computable (there exists a machine $M_s$ computing it).
  Then so is the function $t$, given by $t(k) = s(k) + 1$.
  Indeed, modify $M_s$ to move left, print a stroke and halt instead of halting.
  But no Turing machine can compute $t$: Let $M_t$ be such a machine and $k$ its number of states.
  Then $M_t(k) = t(k) = s(k) + 1 > s(k)$, absurd.
\end{proof}

Busy beaver: a Turing machine starts with a blank tape.
If the machine halts in a standard configuration scanning an $n$-block of $S_1$s on an otherwise blank tape, its productivity is $n$, otherwise $0$.
Now define
\begin{equation*}
  p(n) = \max \left(\text{productivity of any Turing machine with at most $n$ states}\right)
\end{equation*}
This busy-beaver function is also Turing uncomputable.
The following are facts about $p$: $p(1) = 1$, $p(n+1)>p(n)$ for all $n$, there is an $i$ such that $p(n+i) \ge 2p(n)$ for all $n$.

\begin{exercise}[4.1]
  Solution: Alternate rightwards and leftwards from the starting point in steps of increasing length, putting a marker if the destination is blank or halting if finding something that is not the marker.
  If the tape is not completely blank, there has to be a $1$ within a finite length of the initial position.
\end{exercise}

\begin{exercise}[4.2]
  Solution: No such Turing machine exists.
  The tape is infinite in both directions, so no matter how many steps we take in some direction, we can still take that many plus one steps; and we \emph{still} won't know if the tape is completely blank or not, so we can never halt.
\end{exercise}

\begin{exercise}[4.3]
  Copying machine $C$ solution:
  \begin{align*}
    q_1S_0Lq_7,&~~q_1S_1S_0q_2&\text{(go parking if blank or put a marker if not)}\\
    q_2S_0Rq_3,&~~q_2S_1Rq_1&\text{(helper routines)}\\
    q_3S_0Rq_4,&~~q_3S_1Rq_3&\text{(get to the end of the copy)}\\
    q_4S_0S_1q_5,&~~q_4S_1Rq_4&\text{(get to the end of the copy and write a $1$)}\\
    q_5S_0Lq_6,&~~q_5S_1Lq_5&\text{(go back to the marker)}\\
    q_6S_0S_1q_2,&~~q_6S_1Lq_6&\text{(go back to the marker and overwrite it with $1$)}\\
    q_7S_0Rq_8,&~~q_7S_1Lq_7&\text{(parking in the standard final position)}
  \end{align*}
  The idea is to put a blank marker at the start of the string of $1$s, then skip the $1$s until a blank occurs, skip the blank (and $1$s, if any) and write a $1$, move back to the marker, replace it with $1$, move right and put a marker again, and so on until the marker and the blank between the copied strings co-occur.
  When that happens, park in the standard final configuration and we are done.
\end{exercise}

\begin{exercise}[4.4]
  If a two-place function $g$ is Turing computable, then so is the function $f$ given by $f(n) = g(n,n)$.
\end{exercise}
\begin{proof}
  The copying machine $C$ has states $1,\ldots,p$
  Now let $G$ be the Turing machine that computes $g$ with states $1,\ldots,q$.
  Renumber the states of $G$ as $p+1,\ldots,p+q$ and append them after the states of $C$.
  This gives us a Turing machine that copies the input $n$ and applies the function $g$ to those inputs, which computes the function $f$ as desired.
\end{proof}

\begin{exercise}[4.5]
  Let $U$ be a Turing machine (computing a two-place function $u$) such that for any other Turing machine $M_n$ (computing a one-place function $m_n$) and any $x$, we have that $u(n,x) = m_n(x)$.
  This Turing machine $U$ is called a universal Turing machine (UTM).

  If Turing's thesis is correct, then a UTM must exist.
\end{exercise}
\begin{proof}
  \tk See \href{http://math.stackexchange.com/questions/1408692/}{http://math.stackexchange.com/questions/1408692/} and correct the ``solution'' when you get to recursive enumerability and fixed points to understand the answer therein properly.

  The set of Turing machines is enumerable, then for any finite $n$ we can proceed to enumerate the machines as finite strings.
  Once we get to the required $n$, we run the Turing machine $M_n$ on the argument $x$.
  If $M_n$ halts in a standard position with some output, then we have found $u(n,x)$.
  If $M_n$ does not halt or halts in a nonstandard position, then \emph{that} is the desired ``value'' ($u(n,x)$ is then undefined).
  This is an informal list of instructions to compute $u(n,x)$, so $u$ is effectively computable, and so by Turing's thesis is also Turing-computable, whence $U$ must exist.
\end{proof}
% section uncomputability (end)
