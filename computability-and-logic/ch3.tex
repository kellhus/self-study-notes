\section{Turing Computability} % (fold)
\label{sec:turing_computability}
A function $f\colon P^k \to P$, $m_1,\ldots,m_k \mapsto n$ is Turing computable if there is a Turing machine computing it that satisfies the following four conditions:

\textbf{(1) }The arguments $m_1,\ldots,m_k$ are in monadic (tally) notation, each argument separated from the others by a single blank.

\textbf{(2) }Standard initial configuration: the machine is scanning the leftmost $1$ on the tape;

\textbf{(3) }Standard final configuration: the machine halts scanning the leftmost $1$ on the tape containing a block of $n$ strokes, and otherwise blank;

\textbf{(4) }If the function is undefined for the arguments, the machine either never halts or halts in a nonstandard configuration.

Do the first two exercises to remember the notation, if the need later arises.
The rest of the exercises are far too annoying and seemingly mathematically pointless.

\begin{exercise}[3.1]
  Solutions:

  a) $q_1S_1Rq_2, q_1S_0Rq_2$.

  b) $q_1S_1Rq_1, q_1S_0Rq_2$; $q_2S_1Rq_2, q_2S_0Rq_3$.
\end{exercise}

\begin{exercise}[3.2]
  Solutions:

  a) $q_1S_1Rq_1, q_1S_0Rq_2$; $q_2S_1Rq_2, q_2S_0Lq_3$.

  b) $q_1S_1Rq_1, q_1S_0Rq_2$; $q_2S_1Rq_2, q_2S_0Rq_3$; $q_3S_1Rq_3, q_3S_0Lq_4$.
\end{exercise}

\begin{summary}
  \begin{itemize}
    \item A function $f$ from positive integers to positive integers is effectively computable if a list of instructions exists to compute $f(n)$ for any $n$;
    \item A numerical function is Turing computable if a Turing machine computing it exists;
    \item All Turing computable functions are effectively computable.
    \textbf{Turing's thesis} is that any effectively computable function is Turing computable;
    \item Examples of Turing machines in this chapter and its exercises include machines that: write a specified number of strokes, double the number of strokes, determine the parity of the length of a block of strokes, add two blocks of strokes, multiply two blocks of strokes, compute the minimum function, compute the maximum function.
  \end{itemize}
\end{summary}
% section turing_computability (end)
